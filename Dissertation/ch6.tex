\chapter{Investigating the Advancements in CA Research}
\section{Six}
\begin{lstlisting}[language=python, caption=Initial CA Engine]
import numpy as np
from scipy import signal
import matplotlib.pyplot as plt
import matplotlib.animation as animation


"""
Game of life rules:

1. Any live cell with fewer than two live neighbours dies, as if caused by underpopulation.

2. Any live cell with two or three live neighbours lives on to the next generation.

3. Any live cell with more than three live neighbours dies, as if by overpopulation.

"""

class Grid:
    def __init__(self, shape=(10, 10), boundary="fixed"):
        self.shape = shape
        self.boundary = boundary
        self.grid = self.generate_grid()
        self.KERNEL = np.asarray([[1, 1, 1], [1, 0, 1], [1, 1, 1]])  # The KERNEL to convolve cells

    def __repr__(self) -> str:
        return "\n" + str(self.grid).replace("[", " ").replace("]", " ") + "\n"

    def generate_grid(self) -> np.ndarray:
        return np.zeros(shape=self.shape, dtype=np.int8)

    def define_1(self, pos_x, pos_y) -> None:
        self.grid[pos_y][pos_x] = 1

    def define_0(self, pos_x, pos_y) -> None:
        self.grid[pos_y][pos_x] = 0

    @staticmethod
    def growth(U):
        return 0 + (U == 3) - ((U < 2) | (U > 3))

    def next_gen(self):  # Determine a cell's state by it's neighbour count
        self.grid = np.clip(
            self.grid + self.growth(signal.convolve2d(self.grid, self.KERNEL, mode='same', boundary='wrap')), 0, 1)


if __name__ == "__main__":
    frames = []
    grid = Grid(shape=(50, 50))
    grid.define_1(1, 0)
    grid.define_1(2, 0)
    grid.define_1(0, 2)
    grid.define_1(1, 2)
    grid.define_1(2, 2)
    for i in range(100):
        grid.next_gen()
        frames.append(grid.grid)
    fig, ax = plt.subplots()
    ax.set_title("Game of Life")
    print(frames)
    Writer = animation.writers['ffmpeg']
    writer = Writer(fps=15, metadata=dict(artist='Me'), bitrate=1800)


    def update_frame(num):
        ax.imshow(frames[num], cmap='Blues')

    ani = animation.FuncAnimation(fig, update_frame, frames=range(len(frames)), repeat=True)
    ani.save('video.mp4', writer=writer)

\end{lstlisting}
\Gls{Kernel}
